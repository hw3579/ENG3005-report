
\section{Conclusions}
\FloatBarrier % Now figures cannot float above section title

Based on the presented graphs, the pressure distribution over the aerofoil for both the upper and lower surfaces demonstrates significant variations across different chord positions. The pressure coefficients on the upper surface tend to have more negative values as compared to the lower surface, indicating a higher velocity of flow and consequently a lower pressure. This is a fundamental concept in aerodynamics that contributes to lift generation.

Furthermore, the "CL vs. Angle" graph showcases the lift coefficient's relationship with the angle of attack. An initial increase in the lift coefficient is observed as the angle of attack rises. However, after reaching a peak, the lift coefficient appears to plateau or slightly decrease. This suggests that the aerofoil reaches its optimal performance at a specific angle, after which increasing the angle does not yield additional lift and may even lead to stall conditions.

In conclusion, the pressure distribution patterns on the aerofoil validate the principles of aerodynamics and the generation of lift. The relationship between lift coefficient and angle of attack is crucial in understanding the performance and efficiency of the aerofoil, indicating the importance of optimizing the angle for maximum lift and avoiding angles that could lead to diminished performance or stalling.

About error analysis, as the degree increases, the error percentages for both Simpson's Rule and Romberg's method generally decrease, indicating that the accuracy of these integration methods improves at higher degrees. In order to get more accurate results, we need to conduct more experiments to reduce the error and calculate more accurate values.

