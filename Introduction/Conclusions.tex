
\section{Conclusions}
\FloatBarrier % Now figures cannot float above section title

Based on the \autoref{cp} the pressure distribution over the aerofoil for both the upper and lower surfaces demonstrates significant variations across different chord positions. The pressure coefficients on the upper surface tend to have more negative values as compared to the lower surface, indicating a higher velocity of flow and consequently a lower pressure. Furthermore, the \autoref{tcl} showcases the lift coefficient's relationship with the angle of attack. An initial increase in the lift coefficient is observed as the angle of attack rises. However, after reaching a peak, the lift coefficient appears to plateau or slightly decrease.

In conclusion, the pressure distribution patterns on the aerofoil validate the principles of aerodynamics and the generation of lift. The relationship between lift coefficient and angle of attack is crucial in understanding the performance and efficiency of the aerofoil, indicating the importance of optimizing the angle for maximum lift and avoiding angles that could lead to diminished performance or stalling.

About error analysis, as the degree increases, the error percentages for both Simpson's Rule and Romberg's method generally decrease, indicating that the accuracy of these integration methods improves at higher degrees. In order to get more accurate results, we need to conduct more experiments to reduce the error and calculate more accurate values.

\textbf{Question in handbook}

$\bullet$ Can you explain why the aerofoil has to be thin?

Calculation of $p_{eff}$ is based on the pressure at the outlet and the head difference (85/135mm) in \autoref{Peff}. Since $p_{eff}$ is sufficient for thin, the default air flow cross-sectional area is equal to the outlet area. When the peff has a certain thickness, the area difference factor needs to be considered, which has an impact on the experimental results.

$\bullet$ Near stall, the manometer readings may fluctuate significantly; why is this?

One possible reason is flow Separation and reattachment. As the angle of attack approaches the stalling angle, the flow begins to separate from the upper surface of the aerofoil which causing the pressure distribution on the surface to fluctuate. Another reason is turbulent flow. When the angle of attack approaches the stalling angle, it becomes highly turbulent which can lead to fluctuating pressure readings.

