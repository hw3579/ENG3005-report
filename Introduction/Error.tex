\subsection*{Error analysis}
Error analysis is very important which means the systematic evaluation of the uncertainties or inaccuracies in measurements and calculations, helping to understand the sources and implications of potential deviations from true values.
The possible errors in the experiment are:
\begin{enumerate}
    \item Mercury density and air density are not accurate.
    \item The trapezoidal rule is using discrete point to evaluate the integral, which occur the error.
    \item Pitot tube not sealing well and leaking air.
\end{enumerate}
Some examples of potential errors are given above. The first and third of these are systematic errors and the second is a random error.
Systematic errors can only be improved by improving the accuracy of the measuring equipment. Next, the focus is on analysing the random errors due to the trapezoidal integration method.
The trapezoidal integration method uses a linear connection for integration, which can be optimised by first connecting the discrete points with a smooth curve and then using the different integration method to approximate the true value. The results is in \autoref{t5.5}.    \footnote{The error percentages are calculated based on the values of the Trapezoidal method.}

\begin{table}[htbp]
    \caption{Different integration method to calculate $C_L$}  
    \label{t5.5}
    \centering
    \resizebox{0.8\textwidth}{!}{
    \begin{tabular}{@{}ccccccccc@{}}\toprule
    Degree & 0 & 5 & 10 & 15 & 17.5 & 20 & 22.5 & 25 \\\midrule
    Trapezoidal & -0.040 & 0.383 & 0.676 & 0.939 & 1.015 & 0.840 & 0.828 & 0.830\\
    Simpson's Rule & -0.0206 & 0.4612 & 0.8561 & 1.1218 & 1.2110 & 0.9162 & 0.8964 & 0.8975\\
    Error  (Simpson's) & 48.5\% & 20.4\% & 26.6\% & 19.5\% & 19.4\% & 9.1\% & 8.3\% & 8.1\% \\
    Romberg's method & -0.0228 & 0.4606 & 0.8634 & 1.1261 & 1.2179 & 0.9181 & 0.8989 & 0.9006\\
    Error  (Romberg's) & 43.0\% & 20.3\% & 27.7\% & 19.9\% & 20.0\% & 9.3\% & 8.5\% & 8.5\% \\\bottomrule
    \end{tabular}}
    \end{table}

    \vspace{2cm}