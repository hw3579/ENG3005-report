\section{Method}
\FloatBarrier % Now figures cannot float above section title

a. An accurate value for the density of air was determined using room temperature and atmospheric pressure.

b. With the aerofoil positioned at a 0° angle of attack, pressure measurements were recorded.

c. The effective static pressure should be computed by using the atmospheric and duct inlet pressures.

d. Using the effective static pressure ($p_{eff}$) and the airbox pressure reading, calculate the free stream velocity and determined the Reynolds number with chord length. 

e. For each pressure tapping reading, the corresponding value of pressure ratio($C_{p,n}$) was calculated, and these values were plotted against ($\frac{x}{c}$). The curves were extended to represent a chord ratio of 0 and 1. Noticed that the pressure coefficient near the leading edge is approaching zero, which indicate the directing of the air is disappeared at this point. The exact position of this stagnation point changes with incident angle.

f. From the generated curves of $C_p$, the lift coefficient ($C_L$)can be numerically integrated to derive its value.

g. The measurements have been repeated for increasing angles of attack (in steps of $5^\circ$) up to $25^\circ$, with additional measurements at $17.5^\circ$ and $22.5^\circ$.
Following this, the lift coefficients were computed and a graph illustrating $C_L$ vs. $\alpha$ for the aerofoil was plotted.